\documentclass[10pt]{article}

\usepackage{enumerate}
\usepackage[width=18cm,left=1.5cm,right=1.5cm,height=25cm,top=1.6cm,bottom=1cm]{geometry}

\setlength{\parindent}{0pt}
\pagestyle{empty}

\begin{document}

\textbf{\Large Luca Bertagna}

\hrulefill

\vspace*{0.3cm}
\begin{tabular}{p{2.5cm}|p{15.5cm}}
\textsc{Contact\linebreak Information} &

\begin{tabular}{p{10cm}p{5.5cm}}
Sandia National Laboratories & \verb|mobile: 404-660-5493|\\
Computational Sciences (1446) & \verb|e-mail:lbertag@sandia.gov|\\
1450 Innovation Parkway SE, Albuquerque, NM, 87123 & \verb|luca.bertagna84@gmail.com|
\end{tabular}
\vspace*{0.1cm}
\\
\hline
\\

\textsc{Research\linebreak Highlights} & My interests are in the field of applied mathematics, with particular emphasis on modeling and simulating phenomena arising from many engineering problems and applied sciences in general. My research area includes numerical methods for PDE's, computational fluid dynamics, C/C++ programming, HPC, optimization and data assimilation.

\\
\textsc{Education} &

\textbf{Emory University}, Atlanta, GA \hspace*{6cm} \verb|www.mathcs.emory.edu|
\textit{PhD in Applied Mathematics} \hspace*{5.5cm}\textbf{August 2009 - December 2015}
Thesis: \textit{Reliable direct and inverse methods in computational hemodynamics}.

\vspace*{0.3cm}

\textbf{Politecnico di Milano}, Milano (Italy) \hspace*{4.5cm} \verb|www.mate.polimi.it/?lg=en|
\textit{Master's degree in Mathematical Engineering} \hspace*{3cm}\textbf{September 2006 - April 2009}
Thesis: \textit{Models for ion acceleration via ultra-short and ultra-intense laser pulses}.

\vspace*{0.3cm}

\textbf{Politecnico di Milano}, Milano (Italy) \hspace*{4.5cm} \verb|www.mate.polimi.it/?lg=en|
\textit{Bachelor's degree in Mathematical Engineering} \hspace*{2.2cm}\textbf{September 2003 - September 2006}
Thesis: \textit{The Lattice Boltzmann Method and its application in the simulation of 2D flows}.

\\
\textsc{Employment} &

\textbf{Sandia National Laboratories}, Albuquerque, NM\\
& \textit{Research \& Development (LTE)} \hspace*{6.5cm} \textbf{October 2018 - present}

\\
&\textbf{Sandia National Laboratories}, Albuquerque, NM\\
& \textit{Postdoctoral Appointee} \hspace*{6.7cm} \textbf{October 2016 - October 2018}

\\
&\textbf{Florida State University}, Tallahassee, FL\\
& \textit{Postdoctoral Research Associate} \hspace*{5.1cm} \textbf{February 2015 - October 2016}

\\
&\textbf{Emory University}, Atlanta, GA\\
& \textit{Teaching assistant} \hspace*{7.2cm} \textbf{August 2009 - December 2014}

\\
&\textbf{Lawrence Livermore National Laboratory}, Livermore, CA\\
& \textit{Student Internship} \hspace*{8cm} \textbf{June 2014 - August 2014}

\\
\textsc{Teaching}
 & \textbf{Emory University}, Atlanta, GA\\
 & \textit{Instructor for calculus and programming classes} \hspace*{2.5cm} \textbf{August 2010 - December 2014}

\\
\textsc{Languages} & Italian (native), English (fluent), Spanish (basic).

\\
\textsc{Computer\linebreak Skills} &

Operating systems: Linux, Windows.

Text editing/formatting: Latex, Office, Vi.

Programming languages: C/C++ (excellent), Fortran (good).

Scientific Computing libraries and tools: LifeV, Trilinos, Albany, Matlab, Paraview

% \textsc{Relevant programming projects} & $\bullet$ Main contributions within the LifeV library:

% - Implementation of deconvolution-based nonlinear Leray solver.

% - Implementation of fluid-membrane interaction solver.

% - Implementation of quadratic meshes and basic differential geometry structures on manifolds.

% $\bullet$ Main contributions within the Albany library:

% - Inside the FELIX module: sliding boundary condition for ice sheet, subglacial hydrology.

% - General purpose: side discretizations and manifold equations.

% $\bullet$ Scientific computing project during the Master's degree: Implementation in C++ of a Schwartz domain decomposition solver with arbitrary degree spectral elements for variable coefficients elliptic PDE's on quadrilateral 2D-meshes.

\end{tabular}

\newpage
\begin{tabular}{p{2.5cm}|p{15.5cm}}
\textsc{Publications} & \textbf{Peer-reviewd journal articles}
\vspace*{0.2cm}

M.J.Hoffman, M.Perego, S.F.Price, W.H.Lipscomb, T.Zhang, D.Jacobsen, I.K.Tezaur, A.G.Salinger, R.Tuminaro, L.Bertagna, \textit{MPAS-Albany Land Ice (MALI): a variable-resolution ice sheet model for Earth system modeling using Voronoi grids}, Geoscientific Model Development, Vol 11, 2018
\vspace*{0.2cm}

L.Bertagna, M.Deakin, O.Guba, D.Sunderland, A.M.Bradley, I.K.Tezaur, M.A.Taylor, A.G.Salinger \textit{HOMMEXX 1.0: A Performance Portable Atmospheric Dynamical Core for the Energy Exascale Earth System Model}, Geoscientific. Model Development, Vol 12, 2019.
\vspace*{0.2cm}

L.Bertagna, M. Gunzburger, \textit{Well posedness of a coupled ice-hydrology problem arising in glaciology}, SIAM Journal on Mathematical Analysis, Vol 49, Issue 2,2017.
\vspace*{0.2cm}

L.Bertagna, A.Quaini, A.Veneziani, \textit{Deconvolution-based nonlinear filtering for incompressible flows at moderately large Reynolds numbers}, International Journal for Numerical Methods in Fluids, Vol 81, Number 8, 2016.
\vspace*{0.2cm}

L.Bertagna, A.Veneziani, \textit{A model reduction approach for the variational estimation of vascular compliance by solving an inverse fluid-structure interaction problem}, Inverse Problems, Vol 30, Number 5, (2014).
\vspace*{0.2cm}

M. Passoni, L. Bertagna, A. Zani, \textit{Target normal sheath acceleration: theory, comparison with experiments and future perspectives}, New Journal of Physics, Vol 12 (2010).
\vspace*{0.2cm}

M.Passoni, L. Bertagna, A. Zani, \textit{Energetic ions from next generation ultraintense ultrashort lasers: Scaling laws for Target Normal Sheath Acceleration}, Nuclear Instruments and Methods in Physics Research Section A, Vol 620, (2010).
\vspace*{0.2cm}

\textbf{Conference proceedings}

\vspace*{0.2cm}
M.Passoni, L. Bertagna, A. Zani, \textit{Energetic ions from next generation ultraintense ultrashort lasers: Scaling laws for Target Normal Sheath Acceleration}. AIP Conference Proceedings; 2/2/2010, Vol. 1209 Issue 1, p87.

\vspace*{0.2cm}
\textbf{Chapters in books}

\vspace*{0.2cm}
L.Bertagna, A.Quaini, L.Rebholz, A.Veneziani, \textit{On the sensitivity to the filtering radius in Leray models of incompressible flow}, in ``Contributions to Partial Differential Equations and Applications'', (B. N. Chetverushkin et al. eds.), Springer (July 2018).

\vspace*{0.2cm}
L.Bertagna, M.D'Elia, M.Perego, A.Veneziani, \textit{Data Assimilation in Computational Hemodynamics}, in ``Fluid-Structure Interaction and Biomedical Applications'', (B.Tomáš et al. eds.), Springer (July 2014).
\vspace{0.2cm}
\\
\end{tabular}

\begin{tabular}{p{2.5cm}|p{15.5cm}}
\\
\textsc{References} & \textbf{Max Gunzburger}, Frances Eppes Eminent Professor and Chair, Department of Scientific Computing, Florida State University, Tallahassee (FL). e-mail: \verb|mgunzburger@fsu.edu|.

\vspace*{0.1cm}
\textbf{Alessandro Veneziani}, Professor, Department of Mathematics and Computer Science, Emory University, Atlanta (GA). e-mail: \verb|ale@mathcs.emory.edu| (PhD advisor).

\vspace*{0.1cm}
\textbf{Michele Benzi}, Samuel Candler Dobbs Professor, Department of Mathematics and Computer Science, Emory University, Atlanta (GA). e-mail: \verb|benzi@mathcs.emory.edu|.
\end{tabular}
\end{document}
